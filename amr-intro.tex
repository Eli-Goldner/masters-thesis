An AMR represents the meaning of a sentence with a single-rooted, directed, acyclic graph with nodes labeled with concepts and edges labeled with relations. The primary component of an AMR is the predicate-argument structure, with the predicate being a concept that takes a number of arguments as their children. The predicate and its arguments are represented as nodes in the AMR graph,  and the edges represent the relation between the predicate and each of its arguments. The PENMAN notation and graph below represent the AMR for ``John can't afford a car at the moment'': 

\enumsentence{
a. \small\texttt{(p / possible-01\\
\hspace*{1.0cm}:ARG0 (a / afford-01\\
\hspace*{1.5cm}:ARG0 (p / person\\
\hspace*{2.0cm}:name (n / name\\
\hspace*{2.5cm}:op "John"))\\
\hspace*{1.5cm}:ARG1 (c /car)\\
\hspace*{1.5cm}:time (m / moment))\\
\hspace*{1.0cm}:polarity -)}
\\\\
b. \begin{tikzpicture}[node distance=0.9cm, auto,]
 \node[amrnode] (possible) {possible-01};
 \node[amrnode, below left=0.9cm of possible] (neg) {-};
 \node[amrnode, below=0.9cm of possible] (afford) {afford-01};
 \node[amrnode, below left=0.9cm of afford] (person) {person};
 \node[amrnode, below =0.9cm of afford] (car) {car};
 \node[amrnode, below right=0.9cm of afford] (moment) {moment};
 \node[amrnode, below =0.9cm of person] (name) {name};
 \node[amrnode, below =0.9cm of name] (john) {``John''};
 \path (possible) edge[amrarrow] node[auto] {ARG0} (afford)
       (possible) edge[amrarrow] node[above left] {polarity} (neg)
       (afford) edge[amrarrow] node[above left] {ARG0} (person)
       (afford) edge[amrarrow] node[auto] {ARG1} (car)
       (afford) edge[amrarrow] node[auto] {time} (moment)
       (person) edge[amrarrow] node[left] {name} (name)
       (name) edge[amrarrow] node[left] {op} (john);
\end{tikzpicture}
}        Hi Bert, can you talk/????

                     

The predicates in an AMR are sense-disambiguated. In the example above, ``possible-01'' refers to the first sense of ``possible'' while ``afford-01'' represents the first sense of ``afford''. A predicate can take a number of core arguments (\em{ARG0}, \em{ARG1}, etc.) as well as adjunct arguments (e.g., {\it time}). 
The semantic roles for the core arguments are defined with respect to each sense of a predicate and they are drawn from the PropBank frame files \footnote{https://verbs.colorado.edu/verb-index}. For example, the semantic roles for the core arguments of different senses of ``afford'' as defined as follows:

\eenumsentence{
%\setlength\itemsep{-5pt}
\item afford-01: be able to spare, have the financial means
\\ {\tt\small ARG0:} haver of financial means, agent \\
  {\tt\small ARG2:} costly thing, theme

\item afford-02: provide, make available
\\
 {\tt\small ARG0:} provider, agent
\\
{\tt\small ARG1:} provided, theme
\\
{\tt\small ARG2:} recipient
 
}


The attraction of AMR-style representation and annotation is the adoption of a {\it predicative core} element along with its arguments: e.g., an event and its participants. This, in turn, leads to an event-rooted graph that has many advantages for parsing and matching algorithms. As can be seen from the example, the predicate-argument structure is front and center in AMR, and we consider this to be a strengh of AMR.

However, as it currently stands, AMR does not represent quantification or its interaction with modality and negation. The challenge is to maintain the focus on the predicate-argument structure while also adequately accounting for linguistic phenomena that operate above the level of the core AMR representation, in particular quantification and modality. 