% 
% File acl2019.tex
%
%% Based on the style files for ACL 2018, NAACL 2018/19, which were
%% Based on the style files for ACL-2015, with some improvements
%%  taken from the NAACL-2016 style
%% Based on the style files for ACL-2014, which were, in turn,
%% based on ACL-2013, ACL-2012, ACL-2011, ACL-2010, ACL-IJCNLP-2009,
%% EACL-2009, IJCNLP-2008...
%% Based on the style files for EACL 2006 by 
%%e.agirre@ehu.es or Sergi.Balari@uab.es
%% and that of ACL 08 by Joakim Nivre and Noah Smith

\documentclass[11pt]{article}
%%\usepackage[hyperref]{acl2021}
\usepackage{times}
\usepackage{amsmath}
\usepackage{amssymb}
\usepackage{amsthm}
\usepackage{latexsym}
\usepackage{lingstyle}
\usepackage{graphicx}
\usepackage{tikz}
\usepackage{hyperref}
\usepackage{bussproofs}
\usepackage{multicol}
%%\usepackage{prftree.sty} 
\usepackage{natbib}
\usepackage{cite}
\usepackage{float}
\usepackage{turnstile}
\usepackage{rotating}

\usepackage[margin=0.5in]{geometry}

\newtheorem{theorem}{Theorem}[section]
\newtheorem{lemma}[theorem]{Lemma}
\newtheorem{prop}{Proposition}
\newtheorem{corollary}{Corollary}


\theoremstyle{definition}
\newtheorem{definition}{Definition}[section]

\theoremstyle{remark}
\newtheorem*{remark}{Remark}

\usetikzlibrary{arrows,positioning,shapes} 
\tikzset{
    %Define standard arrow tip
    >=stealth',
    %Define style for boxes
    amrnode/.style={
             ellipse,
             draw=black, very thick,
             font=\small},
    scopenode/.style={
             ellipse,
             draw=red, very thick, dashed,
             text=red,
             font=\small},
    % Define arrow style
    amrarrow/.style={
            ->,
            thick,
            font=\small},
    scopearrow/.style={
              ->,
              red,
              thick,
              dashed,
              font=\small}
}



 \def\drs#1#2{\begin{tabular}{|l|}\hline #1 \\ \hline \\
                [-8pt] #2\\[-8pt] \\ \hline \end{tabular} }

 \def\ddrs#1{\begin{tabular}{||c||}\hline \\
                [-8pt] #1\\[-8pt] \\ \hline \end{tabular} }

 \def\topdrs#1#2{\begin{tabular}{|l|}\hline #1 \\ \hline \\
                [-8pt] #2 \\[-8pt] \\ \hline \end{tabular} }

 \def\proof#1#2#3#4{\drs{#1}{#2} \ $\vdash$ \ \drs{#3}{#4}}
 \def\modimp#1#2#3#4{\mbox{\drs{#1}{#2} \ $\Box$ \ \drs{#3}{#4}}}
 \def\imp#1#2#3#4{\drs{#1}{#2} \ $\Rightarrow$ \ \drs{#3}{#4}}
 \def\dis#1#2#3#4{\mbox{\drs{#1}{#2} \ $\vee$ \ \drs{#3}{#4}}}
 \def\int#1#2{\mbox{$^{\wedge}$ \ \drs{#1}{#2}}}
 \def\pos#1#2{\mbox{$\Diamond$ \ \drs{#1}{#2}}}
 \def\nec#1#2{\mbox{$\Box$ \ \drs{#1}{#2}}}
 \def\nega#1#2{\mbox{$\neg$ \ \drs{#1}{#2}}}
 \def\pred#1#2#3{\mbox{#1\ :\drs{#2}{#3}}}


\usepackage{url} 

\usepackage[acronym]{glossaries} % 'nomain' to disable automatic generation of "glossary" section 
\glsdisablehyper % disable hyperlink to non-existing glossary section 
%\aclfinalcopy % Uncomment this line for the final submission
%\def\aclpaperid{***} %  Enter the acl Paper ID here

%\setlength\titlebox{5cm}
% You can expand the titlebox if you need extra space
% to show all the authors. Please do not make the titlebox
% smaller than 5cm (the original size); we will check this
% in the camera-ready version and ask you to change it back.

\newcommand\BibTeX{B\textsc{ib}\TeX}

\title{Formalization of AMR Inference via Hybrid Logic Tableaux}
 

 
\author{
  Eli Goldner
}
\begin{document}
\maketitle
\begin{abstract}
  AMR and its extensions have become popular in semantic representation due
  to their ease of annotation by non-experts, attention to the predicative core of sentences,
  and abstraction away from various syntactic matter.
  An area where AMR and its extensions warrant improvement is formalization
  and suitability for inference, where it is lacking compared to
  other semantic representations, such as
  description logics, episodic logic, and discourse representation theory.
  This thesis presents a formalization of inference over a merging of
  \citeauthor{donatelli-etal-2018-annotation}'s \citeyearpar{donatelli-etal-2018-annotation} AMR extension for tense and aspect
  and with \citeauthor{pustejovsky-etal-2019-modeling}'s \citeyearpar{pustejovsky-etal-2019-modeling} AMR extension for quantification and scope.
  Inference is modeled with a merging of Blackburn and Marx's tableau method for quantified hybrid logic
  ({\it QHL}) and Blackburn and J{\o}rgensen's tableu method for basic hybrid tense logic ({\it BHTL}).
  We motivate the merging of these AMR variants,
  present their interpretation and inference in the combination of {\it QHL}
  and {\it BHTL}, which we will call {\it QHTL} (quantified hybrid tense logic),
  and demonstrate {\it QHTL}'s the soundess, completeness,
  and decidability. 


  
\end{abstract}


\section{Merging Quantified Hybrid Logic and Indexical Hybrid Tense Logic}

\subsection{Background}

\subsection{Quantified Hybrid Logic}

\subsection{Basic Hybrid Tense Logic}

\section{Quantified Hybrid Tense Logic - Syntax and Semantics}
\smallskip
The syntax of $QHTL$ is identical to $QHL$ except
uses of $\downarrow$ as in $\downarrow w . \phi$ are omitted along
with $\Box$ and $\Diamond$ as in $\Box \phi$ and $\Diamond \phi$.
$\Box$ and $\Diamond$ are replaced by their semantic equivalents
$F$ and $G$ and their temporal duals $P$ and $H$ are added.

\smallskip


Atomic formulae are the same as in $QHL$, symbols in \textsf{NOM}
and \textsf{SVAR} together with first-order atomic formulae
generated from the predicate symbols and equality over the terms. 
Thus complex formulae are generated from the atomic formulae according to the
following rules:
$$\neg \phi | \phi \land \psi | \phi \lor \psi | \phi \to \psi | \exists x \phi | \forall x \phi | F \phi | G \phi | P \phi | H \phi | @_n \phi $$



Since we want the domain of quantification to be indexed
over the collection of nominals/times, we alter the $QHL$ model definition
to a structure:
$$(T, R, D_t, I_{nom}, I_t)_{t \in T}$$
Identical to the definition for a $QHL$ model in that:
\begin{itemize}
  \item
    $(T,R)$ is a modal frame.
  \item
    $I_{nom}$ is a function assigning members of $T$ to nominals.
\end{itemize}

The differences manifest on the level of the model and
interpretation.  That is, for every $t \in T$,
$(D_t, I_t)$ is a first-order model where:
\begin{itemize}
\item
  $I_t (q) \in D_t$ where $q$ is a unary function symbol.
\item
  $I_t (P) \subseteq^k D_t$ where $P$ is a $k$-ary predicate symbol.
\end{itemize}

Notice we've relaxed the requirement that $I_t (c) = I_{t'} (c)$ for
$c$ a constant and $t,t' \in T$, since the interpretation of the
constant need not exist at both times.

\smallskip

Free variables are handled similarly as in $QHL$.
A $QHTL$ assignment is a function:
$$g : \textsf{SVAR} \cup \textsf{FVAR} \to T \cup D $$
Where state variables are sent to times/worlds and
first-order variables are sent to $D_t$ where
$t$ is the time assigned to the state variable by $g$.
Thus given a model and an assignment $g$,
the interpretation of terms $t$ denoted by $\overline{t}$
is defined as:
\begin{itemize}
\item
  $\overline{x} = g_t(x)$ for $x$ a variable and the relevant $t \in T$.
\item
  $\overline{c} = I_t (c)$ for $c$ a constant and some $t \in T$.
\item
  \begin{itemize} For $q$ a unary function symbol:
    
    
  \item For $n$ a nominal:
    
    
    $$\overline{@_n q} = I_{I_{nom}} (n)$$
  \item For $n$ a state variable:
    
    
    $$\overline{@_n q} = I_{g(n)} (q)$$
  \end{itemize}
\end{itemize}

With the final adjustment of having $g_{d,s}^x$ denoting
the assignment which is just like $g_s$ except $g_s (x) = d$
for $d \in g(s)$, we can proceed with the inductive definition
for satisfaction of a formula give a model $\mathfrak{M}$,
a variable assignment $g$, and a state $s$.
The inductive definition is:

\begin{alignat*}{2}
  &\mathfrak{M}, g, s \dststile{}{} P(t_1, \dotsc , t_n )  && \Longleftrightarrow \langle \overline{t_1}, \dotsc , \overline{t_n} \rangle \in I_s (P)  \\
  &\mathfrak{M}, g, s \dststile{}{} t_i = t_j  && \Longleftrightarrow \overline{t_i} = \overline{t_j}  \\
  &\mathfrak{M}, g, s \dststile{}{} n && \Longleftrightarrow I_{nom} (n) = s, \text{for }n\text{ a nominal}  \\
  & \mathfrak{M}, g, s \dststile{}{} w && \Longleftrightarrow g(w) = s, \text{for }s\text{ a state variable}  \\
  & \mathfrak{M}, g, s \dststile{}{} \neg \phi && \Longleftrightarrow  \mathfrak{M}, g, s \not\dststile{}{} \phi \\
  & \mathfrak{M}, g, s \dststile{}{} \phi \land \psi && \Longleftrightarrow  \mathfrak{M}, g, s \dststile{}{} \phi \text{ and } \mathfrak{M}, g, s \dststile{}{} \psi \\
  & \mathfrak{M}, g, s \dststile{}{} \phi \lor \psi && \Longleftrightarrow  \mathfrak{M}, g, s \dststile{}{} \phi \text{ or } \mathfrak{M}, g, s \dststile{}{} \psi \\
  & \mathfrak{M}, g, s \dststile{}{} \phi \to \psi && \Longleftrightarrow  \mathfrak{M}, g, s \dststile{}{} \phi \text{ implies } \mathfrak{M}, g, s \dststile{}{} \psi \\
  & \mathfrak{M}, g, s \dststile{}{} \exists x \phi && \Longleftrightarrow \mathfrak{M}, g_{d,s}^x, s \dststile{}{} \phi \text{ for some } d \in D_s \\
  & \mathfrak{M}, g, s \dststile{}{} \forall x \phi && \Longleftrightarrow \mathfrak{M}, g_{d,s}^x, s \dststile{}{} \phi \text{ for all } d \in D_s \\
  & \mathfrak{M}, g, s \dststile{}{} F \phi && \Longleftrightarrow \mathfrak{M}, g, t \dststile{}{} \phi \text{ for some } t \in T \text{ such that } Rst \\
  & \mathfrak{M}, g, s \dststile{}{} G \phi && \Longleftrightarrow \mathfrak{M}, g, t \dststile{}{} \phi \text{ for all } t \in T \text{ such that } Rst \\
  & \mathfrak{M}, g, s \dststile{}{} P \phi && \Longleftrightarrow \mathfrak{M}, g, t \dststile{}{} \phi \text{ for some } t \in T \text{ such that } Rts \\
  & \mathfrak{M}, g, s \dststile{}{} H \phi && \Longleftrightarrow \mathfrak{M}, g, t \dststile{}{} \phi \text{ for all } t \in T \text{ such that } Rts \\
  & \mathfrak{M}, g, s \dststile{}{} @_n \phi && \Longleftrightarrow \mathfrak{M}, g, I_{nom}(n) \dststile{}{} \phi \text{ for } n \text{ a nominal } \\
  & \mathfrak{M}, g, s \dststile{}{} @_w \phi && \Longleftrightarrow \mathfrak{M}, g, g(w) \dststile{}{} \phi \text{ for } w \text{ a state variable }
  %%\caption{$AT_x$}
\end{alignat*}

\subsection{The Tableau Calculus}

Non-branching rules:

\begin{figure*}
  \begin{multicols}{2}
    \begin{prooftree}
      \AxiomC{$@_s \neg \phi$}
      \RightLabel{$[\neg]$}
      \UnaryInfC{$\neg @_s \phi$}
    \end{prooftree}
    
    \begin{prooftree}
      \AxiomC{$\neg @_s \neg \phi$}
      \RightLabel{$[\neg \neg]$}
      \UnaryInfC{$@_s \phi$}
    \end{prooftree}
  \end{multicols}

  \begin{multicols}{2}
    \begin{prooftree}
      \AxiomC{$@_s @_t \phi$}
      \RightLabel{$[@]$}
      \UnaryInfC{$@_t \phi$} 
    \end{prooftree}
    
    \begin{prooftree}
      \AxiomC{$\neg @_s @_t \phi$}
      \RightLabel{$[\neg @]$}
      \UnaryInfC{$\neg @_t \phi$} 
    \end{prooftree}
  \end{multicols}

  \begin{multicols}{2}
    \begin{prooftree}
      \AxiomC{$@_s F \phi$}
      \RightLabel{$[F]$}
      \UnaryInfC{$@_s F a$}
      \noLine
      \UnaryInfC{$@_a \phi$}
    \end{prooftree}
    
    \begin{prooftree}
      \AxiomC{$@_s P \phi$}
      \RightLabel{$[P]$}
      \UnaryInfC{$@_s P a$}
      \noLine
      \UnaryInfC{$@_a \phi$}
    \end{prooftree}
  \end{multicols}

  \begin{multicols}{2}
    \begin{prooftree}
      \AxiomC{$\neg @_s G \phi$}
      \RightLabel{$[\neg G]$}
      \UnaryInfC{$@_s F a$}
      \noLine
      \UnaryInfC{$\neg @_a \phi$} 
    \end{prooftree}
    
    \begin{prooftree}
      \AxiomC{$@_s H \phi$}
      \RightLabel{$[\neg H]$}
      \UnaryInfC{$@_s P a$}
      \noLine
      \UnaryInfC{$\neg @_a \phi$} 
    \end{prooftree}
  \end{multicols}

  \begin{multicols}{2}
    \begin{prooftree}
      \AxiomC{$@_s P t$}
      \RightLabel{$P$-trans}
      \UnaryInfC{$@_t F s$} 
    \end{prooftree}
    
    \begin{prooftree}
      \AxiomC{$@_s F t$}
      \RightLabel{$F$-trans}
      \UnaryInfC{$@_t P s$}  
    \end{prooftree}
  \end{multicols}
  
  \begin{multicols}{2}
    \begin{prooftree}
      \AxiomC{$@_s \exists x \phi(x)$}
      \RightLabel{$[\exists]$}
      \UnaryInfC{$@_c \phi(c)$} 
    \end{prooftree}
    
    \begin{prooftree}
      \AxiomC{$@_s \neg \forall x \phi(x)$}
      \RightLabel{$[\neg \forall]$}
      \UnaryInfC{$\neg @_c \phi(c)$}
    \end{prooftree}
  \end{multicols}

  \begin{multicols}{2}
    \begin{prooftree}
      \AxiomC{$@_s \forall x \phi(x)$}
      \RightLabel{$[\forall]$}
      \UnaryInfC{$ @_c \phi(c)$}
    \end{prooftree}

    \begin{prooftree}
      \AxiomC{$\neg @_s \exists x \phi(x)$}
      \RightLabel{$[\neg \exists]$}
      \UnaryInfC{$\neg @_c \phi(c)$} 
    \end{prooftree}
  \end{multicols}


  \begin{multicols}{2}
    \begin{prooftree}
      \AxiomC{[$s$ on the branch]}
      \RightLabel{{[ Ref ]}}
      \UnaryInfC{$@_s s$} 
    \end{prooftree}
    
    \begin{prooftree}
      \AxiomC{$@_t s$}
      \RightLabel{{[ Sym ]}}
      \UnaryInfC{$@_s t$}
    \end{prooftree}
  \end{multicols}

  \begin{multicols}{2}
    \begin{prooftree}
      \AxiomC{}
      \RightLabel{{[ $=$-Ref ]}}
      \UnaryInfC{$t = t$} 
    \end{prooftree}
    
    \begin{prooftree}
      \AxiomC{$@_n m$}
      \RightLabel{{[ DD ]}}
      \UnaryInfC{$@_n q = @_m q$}
    \end{prooftree}
  \end{multicols}

  \begin{multicols}{2}
    \begin{prooftree}
      \AxiomC{$@_n (t_i = t_j)$}
      \RightLabel{$@=$}
      \UnaryInfC{$t_i = t_j$} 
    \end{prooftree}
    
    \begin{prooftree}
      \AxiomC{$\neg @_n (t_i = t_j)$}
      \RightLabel{$\neg @=$}
      \UnaryInfC{$\neg (t_i = t_j)$} 
    \end{prooftree}
  \end{multicols}
  
  \caption{Non-Branching Rules.}
  \label{nonbranch}
\end{figure*}

Branching rules:

\begin{figure*}
  \begin{multicols}{2}
    \begin{prooftree}
      \AxiomC{$@_s (\phi \lor \psi)$}
      \RightLabel{$\lor$}
      \UnaryInfC{$@_s \phi \mid @_s \psi$} 
    \end{prooftree}
    
    \begin{prooftree}
      \AxiomC{$\neg @_s (\phi \land \psi)$}
      \RightLabel{$\neg \land$}
      \UnaryInfC{$\neg @_s \phi \mid \neg @_s \psi$}
    \end{prooftree}
  \end{multicols}

  \begin{prooftree}
    \AxiomC{$@_s (\phi \to \psi)$}
    \RightLabel{$\to$}
    \UnaryInfC{$\neg @_s \phi \mid @_s \psi$}
  \end{prooftree}
\end{figure*}


Binary rules:

\begin{figure*}
  \begin{multicols}{2}
    \begin{prooftree}
      \AxiomC{$@_s H \phi$}
      \AxiomC{$@_s P t$}
      \RightLabel{$H$}
      \BinaryInfC{$@_t \phi$} 
    \end{prooftree}
    
    \begin{prooftree}
      \AxiomC{$@_s G \phi$}
      \AxiomC{$@_s F t$}
      \RightLabel{$G$}
      \BinaryInfC{$@_t \phi$}
    \end{prooftree}
  \end{multicols}

  \begin{multicols}{2}
    \begin{prooftree}
      \AxiomC{$@_s P t$}
      \AxiomC{$@_t u$}
      \RightLabel{$P$-bridge}
      \BinaryInfC{$@_t P u$} 
    \end{prooftree}
    
    \begin{prooftree}
      \AxiomC{$@_s F t$}
      \AxiomC{$@_t u$}
      \RightLabel{$F$-bridge}
      \BinaryInfC{$@_t F u$}  
    \end{prooftree}
  \end{multicols}

  \begin{multicols}{2}
    \begin{prooftree}
      \AxiomC{$@_s t$}
      \AxiomC{$@_s \phi$}
      \RightLabel{Nom}
      \BinaryInfC{$@_t \phi$} 
    \end{prooftree}
    
    \begin{prooftree}
      \AxiomC{$@_s t$}
      \AxiomC{$@_t \phi$}
      \RightLabel{$\text{Nom}^{-1}$}
      \BinaryInfC{$@_s \phi$} 
    \end{prooftree}
  \end{multicols}

  \begin{prooftree}
      \AxiomC{$@_s t$}
      \AxiomC{$@_t r$}
      \RightLabel{Trans}
      \BinaryInfC{$@_s r$} 
    \end{prooftree}
\end{figure*}

\subsection{Soundness}
The proof of the soundness of the tableau method
for QHTL is adapted from the proof of
soundness of the tableau method for
$\mathcal{H}(@)$ given in \citet*{internalizing}.

\bigskip

We can observe from the tableau rules that
every formula in a tableau is of the
form  $@_s \phi$ or $\neg @_s \phi$.
We call formulae of these forms
{\it satisfaction statements}.
Give a set of satisfaction statements $\Sigma$
and a tableau rule $R$ we develop the
notion of $\Sigma^+$ as an expansion of $\Sigma$
by $R$ as follows based on the different cases for $R$:
\begin{enumerate}
\item

  If $R$ is {\it not} a branching rule, and $R$ takes
  a single formula as input, and $\Sigma^+$ is the set
  obtained by adding to $\Sigma$ the formulae yielded
  by applying $R$ to $\sigma \in \Sigma$,
  then we say $\Sigma^+$ is the result of expanding
  $\Sigma$ by $R$.  
\item

  If $R$ is a binary rule, and $\Sigma^+$ is the set
  obtained by adding to $\Sigma$ the formulae yielded
  by applying $R$ to $\sigma_1 , \sigma_2  , \sigma_2 \in \Sigma$,
  then we say $\Sigma^+$ is the result of expanding
  $\Sigma$ by $R$.
\item

  If $R$ is a branching rule, and $\Sigma^+$ is the set
  obtained by adding to $\Sigma$ the formulae yielded
  by one of the possible outcomes of applying $R$ to $\sigma_1  \in \Sigma$,
  then we say $\Sigma^+$ is the result of expanding
  $\Sigma$ by $R$.
\item

  If a nominal $s$ belongs to some formula in $\Sigma$,
  then $\Sigma^+ = \Sigma \cup { @a_s s }$ is
  the result of expanding $\Sigma$ by \textsf{Ref}.
\item
  
  If a nominal $s$ belongs to some formula in $\Sigma$,
  then $\Sigma^+ = \Sigma \cup { t = t }$ is
  the result of expanding $\Sigma$ by $=$-\textsf{Ref}.

\end{enumerate}


\begin{definition}[Satisfiable by label]

  Suppose $\Sigma$ is a set of satisfaction statements and
  $\mathfrak{M} = (T, R, D_t, I_{nom}, I_t)_{t \in T}$
  is a standard {\it QHTL} model.
  We say $\Sigma$ is
  {\it satisfied by label} in $\mathfrak{M}$
  under a {\it QHTL} assignment $g$
  if and and only if for all formulae in $\Sigma$:
  \begin{enumerate}
  \item
    If $@_s \phi \in \Sigma$ then
    $\mathfrak{M}, g, I_{nom}(s) \dststile{}{} \phi$
  \item
    If $\neg @_s \phi \in \Sigma$ then
    $\mathfrak{M}, g, I_{nom}(s) \not \dststile{}{} \phi$
    
  \end{enumerate}

  We say $\Sigma$ is {\it satisfiable by label}
  if and only if there is a standard {\it QHTL} model
  and assignment in which it is satisfied by label.

  
\end{definition}

\begin{theorem}[Soundess]
  \label{sound}
  If $\Sigma$ is a set of satisfaction
  statements which is satisfiable by label,
  then for any tableau rule $R$, at least one of the
  sets obtainable by expanding $\Sigma$ by $R$
  is satisfiable by label.  
\end{theorem}


%%\begin{proof}
({\it Proof}).  We prove soundness by induction on the
tableau rules, with particular
attention to ``existential'' rules
as they are called in \citet*{internalizing}
which introduce a new parameter on the branch.
In all cases discussed below let 
$\mathfrak{M} = (T, R, D_t, I_{nom}, I_t)_{t \in T}$
and $g$ be the {\it QHTL} model and assignment
in which $\Sigma$ is satisfiable by label.

\begin{itemize}
\item {\it Non-branching Rules}

  We will take the $\land$ rule
  as an example.  Beginning
  from $@_s \phi \land \psi$ we have:
  $$\mathfrak{M}, g, I_{nom}(s) \dststile{}{} \phi \land \psi$$
  and consequentially 
  $$\mathfrak{M}, g, I_{nom}(s) \dststile{}{} \phi \iff @_s \phi$$
  $$\mathfrak{M}, g, I_{nom}(s) \dststile{}{} \psi \iff @_s \psi$$
  Similarly for their negations if at least one of $@_s \phi \land \psi$
  are not satisfied in $\mathfrak{M}$ under $g$.  
  Thus the results of the application of the $\land$ rule, $@_s \phi$ and $@_s \psi$
  are satisfiable in $\mathfrak{M}$ under $g$ and the expansion of $\Sigma$
  by $\land$ is satisfiable by label.
  \bigskip
  The proof for other non-branching rules are analogous.
\item {\it Binary Rules}
  We will take the $H$ rule
  as an example.  Beginning
  with $@_s H \phi$ and $@_s P t$ we have from the former:
  $$\mathfrak{M}, g, t \dststile{}{} \phi \text{ for all } t \in T \text{ such that } R t I_{nom}(s)$$
  and from the latter:
  $$\mathfrak{M}, g, t' \dststile{}{} t \text{ for some } t' \in T \text{ such that } R t I_{nom}(s)$$
 
  and consequentially since $Rts$ 
  $$\mathfrak{M}, g, I_{nom}(t) \dststile{}{} \phi \iff @_t \phi$$
  Similarly for their negations if at least one of $@_s H \phi$ and $@_s P t$
  are not satisfied in $\mathfrak{M}$ under $g$.
  Thus the result of application of the $H$ rule, $@_t \phi$ is
  satisfiable in $\mathfrak{M}$ under $g$ and the expansion of $\Sigma$
  by $H$ is satisfiable by label.
  \bigskip
  The proof for other binary rules is analogous.
\item {\it Branching Rules}
  We will take the $\lor$ rule
  as an example.  Beginning
  from $@_s \phi \lor \psi$ if it's satisfied we have:
  $$\mathfrak{M}, g, I_{nom}(s) \dststile{}{} \phi \lor \psi$$
  and consequentially at least one of 
  $$\mathfrak{M}, g, I_{nom}(s) \dststile{}{} \phi \iff @_s \phi$$
  or 
  $$\mathfrak{M}, g, I_{nom}(s) \dststile{}{} \psi \iff @_s \psi$$
  Similarly for their negations if $@_s \phi \lor \psi$ is not satisfied
  in $\mathfrak{M}$ under $g$.
  Thus at least one of the results of the application of the $\lor$ rule,
  $@_s \phi$ or $@_s \psi$ or their negations
  are satisfiable in $\mathfrak{M}$ under $g$ and the expansion of $\Sigma$
  by $\lor$ is satisfiable by label.
  \bigskip
  The proof for other branching rules are analogous.
\item {\it ``Existential'' Rules}
  We will take the $\forall$ rule
  as an example.  Beginning
  from $@_s \forall x \phi (x)$ if it's satisfied we have
  (where $s' = I_{nom}(s)$):
  $$\mathfrak{M}, g_{d, s'}^x, s' \dststile{}{} \phi \text{ for all } d \in D_{s'} \iff @_s \phi(t')$$
  Similarly for its negation if $@_s \forall x \phi (x)$ is not satisfied
  in $\mathfrak{M}$ under $g$.  
  In accordance with the constraints for the rule we can select $t'$ to be any
  grounded term on the branch which is also a member of $D_{I_{nom}(s)}$
  Thus the result of the application of the $\forall$ rule,
  or its negations
  are satisfiable in $\mathfrak{M}$ under $g$ and the expansion of $\Sigma$
  by $\forall$ is satisfiable by label.
  \bigskip
  The proof for $\neg \exists$ is analogous.
  
\item {\it Ref rules}
  $t = t$ is a tautology, invariant of model or assignment.
  For $@_s s$, we begin with having $s$ is the branch,
  as result we certainly have
  $$\mathfrak{M}, g, I_{nom}(s) \dststile{}{} I_{nom}(s) \iff @_s s$$
  Thus the expansion of $\Sigma$ by $=-$\textsf{Ref} or \textsf{Ref}
  is satisfiable by label.
\end{itemize}
\qed


\begin{figure*}
  \begin{alignat*}{2}
    & AT_x (p) && := Px \\
    & AT_x (n) && := x = n \\
    & AT_x (w) && := x = w \\
    & AT_x (\neg \phi) && := \langle \lambda x . \neg AT_x (\phi) \rangle (x) \\
    & AT_x (\phi \land \psi) && := \langle \lambda x . AT_x (\phi) \land AT_x (\psi) \rangle (x) \\
    & AT_x(G \phi) && := \langle \lambda x . \forall y (Rxy \to AT_y (\phi)) \rangle (x) \\  
    & AT_x(H \phi) && := \langle \lambda x . \forall y (Ryx \to AT_y (\phi)) \rangle (x) \\
    & AT_x(@_n \phi) && := \langle \lambda x . \forall x (x = n \to AT_x (\phi)) \rangle (x) \\
    & AT_x(P(t_1, \dotsc, t_k)) && := P'(x, t_1, \dotsc, t_k) \\
    & AT_x(t_i = t_j) && := \langle \lambda x . t_i = t_j \rangle(x) \\
    & AT_x (\forall v \phi) && := \langle \lambda x . \forall v AT_x(\phi) \rangle (x)  
    %%\caption{$AT_x^-$}
  \end{alignat*}
  \label{atm}
  \caption{TEST}
\end{figure*}


\begin{figure*}
  \begin{alignat*}{2}
    & AT_x^- (Px) && := p \\
    & AT_x^- (x = n) && := n \\
    & AT_x^- (x = w) && := w \\
    & AT_x^- (\langle \lambda x . \neg \phi \rangle (x)) && := \neg AT_x^- (\phi) \\
    & AT_x^- (\langle \lambda x . \neg \phi \land \psi \rangle (x)) && := AT_x^- (\phi) \land AT_x^- (\psi) \\
    & AT_x^-(\langle \lambda x . \forall y (Rxy \to \phi) \rangle (x)) && := G AT_y^- (\phi) \\  
    & AT_x^-(\langle \lambda x . \forall y (Ryx \to \phi) \rangle (x)) && := H AT_y^- (\phi) \\
    & AT_x^-(\langle \lambda x . \forall x (x = n \to \phi) \rangle (x)) && := @_n AT_x^- (\phi) \\
    & AT_x^-(P'(x, t_1, \dotsc, t_k)) && := P(t_1, \dotsc, t_k) \\
    & AT_x^-(\langle \lambda x . t_i = t_j \rangle(x)) && := t_i = t_j \\
    & AT_x^- (\langle \lambda x . \forall v  \phi \rangle (x)) && := \forall v AT_x^- (\phi)   
    %%\caption{$AT_x^-^-$}
  \end{alignat*}
  \label{atm}
  \caption{TEST}
\end{figure*}

\begin{figure*}
  \begin{alignat*}{2}
    & P(t)^* && := @_t p \\
    & (t = u)^* && := @_t u \\
    & (Rst)^* && := @_s F t \\
    & (\langle \lambda x.\phi \rangle (t))^* && := @_t AT_x^- (\langle \lambda x . \phi \rangle (x))\\  
    & (\langle \lambda y.\phi \rangle (t))^* && := @_t AT_y^- (\langle \lambda y . \phi \rangle (y))\\
    & (AT_x(\phi)[t / x])^* && := @_t \phi \\
    & (\neg AT_x(\phi)[t / x])^* && := \neg @_t \phi \\
    & P'(s, t_1, \dotsc, t_k)^* && := @_s P(t_1, \dotsc, t_k) \\
    & (t_i = t_j)^* && := t_i  t_j 
    %% \caption{$AT_x^-^-$}
  \end{alignat*}
  \label{atm}
  \caption{TEST}
\end{figure*}


\subsection{Decidability}

The proof of the tableau construction algoritheorem's termination is adapted from
the proof given in \citet*{bolander} for the termination of the
tableau construction algoritheorem for $\mathcal{H}(@)$
as described in \citet*{internalizing} except extended with the universal modality.

\begin{lemma}[Quasi-subformula property]
\end{lemma}

\begin{definition}
\end{definition}

\begin{theorem}
\end{theorem}

\begin{corollary}
\end{corollary}

\begin{definition}
\end{definition}

\begin{theorem}
\end{theorem}

\begin{definition}
\end{definition}

\begin{prop}
\end{prop}

\begin{definition}
\end{definition}

\begin{definition}
\end{definition}

\begin{theorem}
\end{theorem}

 
\bibliographystyle{acl_natbib}
\bibliography{references}
\end{document}
