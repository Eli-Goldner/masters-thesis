\documentclass[usenames,dvipsnames,pdf]{beamer}

\usepackage{textcomp}
\usepackage{pifont}
\usepackage[utf8]{inputenc}
\usepackage{amsfonts}
\usepackage{amstext}
\usepackage{amsmath}
\usepackage{fancyhdr}
\usepackage{amsthm}
\usepackage{epsfig}
\usepackage{graphicx}
\usepackage{multicol}
% \usepackage{tikz}
\usepackage{bussproofs}
\usepackage[tableaux]{prooftrees}
\usepackage{mathtools}
\usepackage{scalerel,stackengine}
\usepackage[all]{xy}
% \usetikzlibrary{automata, positioning, shapes, arrows}
% \usepackage[dvipsnames]{xcolor}

\usetheme{CambridgeUS}

%\useoutertheme{miniframes} % Alternatively: miniframes, infolines, split
%\useinnertheme{circles}

\definecolor{UBCblue}{rgb}{0.04706, 0.13725, 0.26667} % UBC Blue (primary)

% \usecolortheme[named=UBCblue]{structure}
% \usecolortheme[named=RoyalBlue]{structure}
\usecolortheme{spruce}
% \usecolortheme{beaver}
%\setbeamercolor{spruce}{fg=cyan!90!black}

\setbeamertemplate{itemize item}{\color{teal}$\blacktriangleright$}
\setbeamertemplate{itemize subitem}{\color{teal}$\blacktriangleright$}


% \newcommand{\newState}[4]{\node[state,#3](#1)[#4]{#2};}
% \newcommand{\newTransition}[4]{\path[->] (#1) edge [#4] node {#3} (#2);} 
\renewcommand*\linenumberstyle[1]{(#1)}
\def\apeqA{\SavedStyle\sim}
\def\apeq{\setstackgap{L}{\dimexpr.5pt+1.5\LMpt}\ensurestackMath{%
  \ThisStyle{\mathrel{\Centerstack{{\apeqA} {\apeqA}}}}}}

\def\dis{\displaystyle}

\def\QQ{\mathbb Q}
\def\ZZ{\mathbb Z}
\def\RR{\mathbb R}
\def\CC{\mathbb C}
\def\FF{\mathbb F}
\def\NN{\mathbb N}
\def\AA{\mathbb A}
\def\II{\mathbb I}

\def\Cc{\mathcal C}
\def\Dd{\mathcal D}
\def\Pp{\mathcal P}

\def\Af{\mathfrak A}
\def\Bf{\mathfrak B}
\def\Cf{\mathfrak C}
\def\Df{\mathfrak D}
\def\Ef{\mathfrak E}
\def\Ff{\mathfrak F}
\def\Gf{\mathfrak G}
\def\Hf{\mathfrak H}
  
% define 2x2 matrix:
\newcommand\twodmatrix[4]{ \ensuremath{ \left( 
	\begin{array}{cc}
		#1 & #2  \\
		#3 & #4 
	\end{array}  
	\right) } }
  

%%%%%%%%%%%%%%%%%%%%%%%%%%%%%%%%%%%%%%%%%%%%%

\mode<presentation>{}
%% preamble
\title{Formalizing AMR Inference via Hybrid Logic Tableaux}
\subtitle{CL Masters Thesis Defense}
\author{Eli Goldner}
\begin{document}
	%% title frame
	\begin{frame}
		\titlepage
	\end{frame}


        \section{Overview}

        
        
        \section{AMR Extensions}

        \section{{\it FHTL}}

        \section{Tableaux}

        \begin{frame}{{\it FHTL} Tableau Example} 
          \begin{tableau}
            [{[f(a,b) = f(b,a)]}
              [@_s (\exists x) {[P((\exists y) {[f(x,y) = f(y,x)]})  \lor \neg (\exists z) {[x = z]}]}
                [@_s P((\exists y) {[f( s_1,y) = f(y, s_1)]})  \lor \neg (\exists z) {[s_1 = z]}
                  [@_s P( (\exists y) {[ f( s_1,y) = f(y, s_1) ]} )
                    [@_s P t
                      [@_t (\exists y) {[ f( s_1,y) = f(y, s_1) ]}
                        [\ldots]
                      ]
                    ]
                  ]
                  [@_s \neg (\exists z) {[s_1 = y]}
                    [@_s \neg {[s_1 = s_1]} , close]
                  ]
                ]
              ]
            ]
          \end{tableau}
        \end{frame}
        
        \section{AMR to {\it FHTL}}

        \section{Conclusion}
      \end{document}
