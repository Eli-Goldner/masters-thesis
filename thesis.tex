


% 
% File acl2019.tex
%
%% Based on the style files for ACL 2018, NAACL 2018/19, which were
%% Based on the style files for ACL-2015, with some improvements
%%  taken from the NAACL-2016 style
%% Based on the style files for ACL-2014, which were, in turn,
%% based on ACL-2013, ACL-2012, ACL-2011, ACL-2010, ACL-IJCNLP-2009,
%% EACL-2009, IJCNLP-2008...
%% Based on the style files for EACL 2006 by 
%%e.agirre@ehu.es or Sergi.Balari@uab.es
%% and that of ACL 08 by Joakim Nivre and Noah Smith

\documentclass[11pt]{article}
%%\usepackage[hyperref]{acl2019}
\usepackage{times}
\usepackage{amsmath}
\usepackage{amssymb}
\usepackage{latexsym}
\usepackage{lingstyle}
\usepackage{graphicx}
\usepackage{tikz}
\usepackage{bussproofs}
\usepackage{natbib}
\usepackage{cite}
\usepackage{hyperref}
\usepackage[margin=0.5in]{geometry}
\usetikzlibrary{arrows,positioning,shapes} 
\tikzset{
    %Define standard arrow tip
    >=stealth',
    %Define style for boxes
    amrnode/.style={
             ellipse,
             draw=black, very thick,
             font=\small},
    scopenode/.style={
             ellipse,
             draw=red, very thick, dashed,
             text=red,
             font=\small},
    % Define arrow style
    amrarrow/.style={
            ->,
            thick,
            font=\small},
    scopearrow/.style={
              ->,
              red,
              thick,
              dashed,
              font=\small}
}



 \def\drs#1#2{\begin{tabular}{|l|}\hline #1 \\ \hline \\
                [-8pt] #2\\[-8pt] \\ \hline \end{tabular} }

 \def\ddrs#1{\begin{tabular}{||c||}\hline \\
                [-8pt] #1\\[-8pt] \\ \hline \end{tabular} }

 \def\topdrs#1#2{\begin{tabular}{|l|}\hline #1 \\ \hline \\
                [-8pt] #2 \\[-8pt] \\ \hline \end{tabular} }

 \def\proof#1#2#3#4{\drs{#1}{#2} \ $\vdash$ \ \drs{#3}{#4}}
 \def\modimp#1#2#3#4{\mbox{\drs{#1}{#2} \ $\Box$ \ \drs{#3}{#4}}}
 \def\imp#1#2#3#4{\drs{#1}{#2} \ $\Rightarrow$ \ \drs{#3}{#4}}
 \def\dis#1#2#3#4{\mbox{\drs{#1}{#2} \ $\vee$ \ \drs{#3}{#4}}}
 \def\int#1#2{\mbox{$^{\wedge}$ \ \drs{#1}{#2}}}
 \def\pos#1#2{\mbox{$\Diamond$ \ \drs{#1}{#2}}}
 \def\nec#1#2{\mbox{$\Box$ \ \drs{#1}{#2}}}
 \def\nega#1#2{\mbox{$\neg$ \ \drs{#1}{#2}}}
 \def\pred#1#2#3{\mbox{#1\ :\drs{#2}{#3}}}


\usepackage{url} 

\usepackage[acronym]{glossaries} % 'nomain' to disable automatic generation of "glossary" section 
\glsdisablehyper % disable hyperlink to non-existing glossary section 
%\aclfinalcopy % Uncomment this line for the final submission
%\def\aclpaperid{***} %  Enter the acl Paper ID here

%\setlength\titlebox{5cm}
% You can expand the titlebox if you need extra space
% to show all the authors. Please do not make the titlebox
% smaller than 5cm (the original size); we will check this
% in the camera-ready version and ask you to change it back.

\newcommand\BibTeX{B\textsc{ib}\TeX}

\title{Formalization of AMR Inference via Hybrid Logic Tableaux}
 

 
\author{
  Eli Goldner
}
\begin{document}
\maketitle
\begin{abstract}
  AMR and its extensions have become popular in semantic representation due
  to their ease of annotation by non-experts, attention to the predicative core of sentences,
  and abstraction away from various syntactic matter.
  An area where AMR and its extensions warrant improvement is formalization
  and suitability for inference, where it is lacking compared to
  other semantic representations, such as
  description logics, episodic logic, and discourse representation theory.
  This thesis presents a formalization of inference over a merging of
  \citeauthor{donatelli-etal-2018-annotation}'s \citeyearpar{donatelli-etal-2018-annotation} AMR extension for tense and aspect
  and with \citeauthor{pustejovsky-etal-2019-modeling}'s \citeyearpar{pustejovsky-etal-2019-modeling} AMR extension for quantification and scope.
  Inference is modeled with a merging of Blackburn and Marx's tableaux method for quantified hybrid logic
  ({\it QHL}) and Blackburn and J{\o}rgensen's tableux method for basic hybrid tense logic ({\it BHTL}).
  We motivate the merging of these AMR variants,
  present their interpretation and inference in the combination of {\it QHL}
  and {\it BHTL}, which we will call {\it QHTL} (quantified hybrid tense logic),
  and demonstrate {\it QHTL}'s the soundess, completeness,
  and decidability. 


  
\end{abstract}


\section{Introduction}

\section{Related Work}

We draw from a number of areas which motivate this approach, namely
designing semantic representations for inferentiability,
the history and goals of AMR and its different annotations,
and hybrid logic with its variants and their accompanying tableaux methods for proof.  

\subsection{Inference in Semantic Representation}

Semantic representation is the task of representing meaning at the sentential
and potentially the discourse levels of language in a formally specifiable way.

\subsection{Discourse Representation Theory}

\subsection{Description Logics}

More expressive than propositional logic, less expressive than
first-order logic.  

A DL models {\it concepts}, {\it roles}, and {\it individuals},
and relationships among them .

DL relates roles and concepts via {\it axioms}, the key modeling concept of DL,
in contrast to frame spacifications in AI which declare and completely define classes \cite{GRAU2008309}.

First DL-based knowledge representation system was KL-ONE \cite{KL-ONE}.

Other 1980s DL systems -- {\it substructural subsumption algorithms},
lower expressivity but polynomial time reasoning \cite[Chapter~3]{van2008handbook}.

Introduction of tableaux based algorithms in 90s allowed greater efficiency on
problems in more expressive DL.

Modern DL \cite{Pellet} \cite{FaCT++} and RacerPro \cite{HHMW12} (from Racer \cite{HaMo01a})
\subsection{Situation Semantics and Related}

\subsubsection{Situation Semantics}

\subsubsection{Type Theory with Records}

\subsubsection{PTT}

\subsubsection{KoS}

\subsection{Episodic Logic}

Episodic Logic is a Montague-style logical form based semantic representation
and knowledge representation, 
with relative strength in semantic expressivity and inferetiability
in comparison to other semantic representation, making it better comparatively better suited
for deep NLU \cite{schubert2015}.  Episodic Logic allows for generalized quantifiers,
lambda abstraction,
reification and modification of sentences and predicates, intensional predicates,
unreliable generalizations, and explicit situational variables \cite{schubert-00}.
Episodic Logic with its inference engine {\sc Epilog} provide a way to capture
a relatively comprehensive range of semantic phenomena compared
to other SR/KRs, in a way which affords inference about semantic data
at a comparable efficiency with automated
inference engines for first order logic \cite{schubert2015}.  It also has an associated knowledge base
{\sc Knext} which is capable of parsing sentences into factoids,
generalizing them (through a process called quantificational sharpening),
and making certain judgements about whether a generalized factoid
is redudant or inconsistent with anything established in the knowledge base.

\subsection{Semantic Features in AMR and Possbility of Inference}

\subsection{Hybrid Logic and Our Chosen Semantic Features}

\section{Merging AMR Annotations}

\subsection{AMR Annotated for Tense and Aspect}

\subsection{AMR Annotated for Scope and Quantification}

\section{Merging Quantified Hybrid Logic and Indexical Hybrid Tense Logic}

\subsection{Background}

\subsubsection{Quantified Hybrid Logic}

\subsubsection{Basic Hybrid Tense Logic}

\subsection{Quantified Hybrid Tense Logic}
\smallskip
The syntax of $QHTL$ is identical to $QHL$ except
uses of $\downarrow$ as in $\downarrow w . \phi$ are omitted along
with $\Box$ and $\Diamond$ as in $\Box \phi$ and $\Diamond \phi$.
$\Box$ and $\Diamond$ are replaced by their semantic equivalents
$F$ and $G$ and their temporal duals $P$ and $H$ are added.

\smallskip


Atomic formulae are the same as in $QHL$, symbols in \textsf{NOM}
and \textsf{SVAR} together with first-order atomic formulae
generated from the predicate symbols and equality over the terms. 
Thus complex formulae are generated from the atomic formulae according to the
following rules:
$$\neg \phi | \phi \land \psi | \phi \lor \psi | \phi \to \psi | \exists x \phi | \forall x \phi | F \phi | G \phi | P \phi | H \phi | @_n \phi $$



Since we want the domain of quantification to be indexed
over the collection of nominals/times, we alter the $QHL$ model definition
to a structure:
$$(T, R, D_w, I_{nom}, I_w)_{w \in W}$$
Identical to the definition for a $QHL$ model in that:
\begin{itemize}
  \item
    $(T,R)$ is a modal frame.
  \item
    $I_{nom}$ is a function assigning members of $T$ to nominals.
\end{itemize}

The differences manifest on the level of the model and
interpretation.  That is, for every $t \in T$,
$(D_t, I_t)$ is a first-order model where:
\begin{itemize}
\item
  $I_t (q) \in D_t$ where $q$ is a unary function symbol.
\item
  $I_t (P) \subseteq^k D_t$ where $P$ is a $k$-ary predicate symbol.


  Notice we've relaxed the requirement that $I_t (c) = I_{t'} (c)$ for
  $c$ a constant and $t,t' \in T$, since the interpretation of the
  constant need not exist at both times.

  \smallskip

  Free variables are handled similarly as in $QHL$.
  A $QHTL$ assignment is a function:
  $$g : \textsf{SVAR} \cup \textsf{FVAR} \to T \cup D $$
  Where state variables are sent to times/worlds and
  first-order variables are sent to $D_t$ where
  $t$ is the time assigned to the state variable by $g$.
  Thus given a model and an assignment $g$,
  the interpretation of terms $t$ denoted by $\overline{t}$
  is defined as:
  \begin{itemize}
  \item
    $\overline{x} = g_t(x)$ for $x$ a variable and the relevant $t \in T$.
  \item
    $\overline{c} = I_t (c)$ for $c$ a constant and some $t \in T$.
  \item
    \begin{itemize} For $q$ a unary function symbol:

      
    \item For $n$ a nominal:


      $$\overline{@_n q} = I_{I_{nom}} (n)$$
    \item For $n$ a state variable:

      
      $$\overline{@_n q} = I_{g(n)} (q)$$
    \end{itemize}
  \end{itemize}
\end{itemize}

With the final adjustment of having $g_{d,s}^x$ denoting
the assignment which is just like $g_s$ except $g_s (x) = d$
for $d \in g(s)$, we can proceed with the inductive definition
for satisfaction of a formula give a model $\mathfrak{m}$,
a variable assignment $g$, and a state $s$.  (Will give this in full soon)  
\subsection{The Tableaux Calculus}


(Sketch)

The main issue with the tableaux of the merged logics is
treatment of the quantification rules,
for the existential rule, the quantifer is removed
and a parameter new on the branch is substituted for the
formerly bound variable, and in the universal case,
the bound variable in the formula is substituted for a term
already grounded on the branch (a first-order consant, parameter,
or grounded definite description).  What is now at issue is unlike
$QHL$ we are not using a fixed domain semantics, thus we must find a
way to integrate the constraint that for universal quantification,
the grounded term needs to have a known interpretation at the current
world/state/time.
\smallskip
NB: Other than the issue of encoding this constraint I see no reason
why the same approach of merging would not work here as well.




\subsection{Soundness and Completeness}
(Sketch)

In either of the cases mentioned below,
the proof of soundess would likely proceed
by checking satisfaction in a way \citet*{blackburn2012indexical}
 refers to being demonstrated in \citet*{internalizing} 
For integrating basic hybrid tense logic
rather than indexical hybrid logic, the completeness proof
seems merely to be an issue of integrating $AT_x$ 
translations $F \phi$, $P\phi$ and $AT_x^-$ translations
of their images under $AT_x$ into the completeness proof of $QHL$
in \citet*{quantified}.
\smallskip
NB: To my current understanding it's less clear how to give a completeness
proof for $QHL$ with full indexical hybrid tense logic, although since the
proof does not seem to make much use of the structure of formulae outside
of tense, it's also not clear to me that adding quantification
would cause many/any issues, and if so the completeness proof would
be adapted mostly from \citet*{blackburn2012indexical}
rather than \citet*{quantified}.    

\subsection{Decidability}
\subsubsection{Decidabililty of the Merged Logic}


(Proof Sketch)


While $\mathcal{H}(\downarrow @)$ is not decidable, $\mathcal{H}(@)$ is
\citep{Areces99hybridlogics}.  Quantified hybrid logic makes use of
$\downarrow$ \citep{quantified}, but modulo $\downarrow$ and quantification over
first-order variables does not
differ from $\mathcal{H}(@)$.
Basic hybrid tense logic does not make use of $\downarrow$
\citep{blackburn2012indexical}, and differs only from
$\mathcal{H}(@)$ in replacing the $\Box$ and $\Diamond$
with $F$ and $P$, which respectively have the same sematics
and similar semantics (the direction of the accesibility relation is changed)
to $\Diamond$.  Similarly for $G$ and $H$ respectively in relation to $\Box$.  Thus given the absence of $\downarrow$ in the merged logic,
replacing $\Box$ and $\Diamond$ with $F$ and $P$ (and by extension $G$ and $H$)
will not have negatively affect the decidability given the analogous
complexity of $F$ and $P$ to $\Diamond$.
\smallskip
This is especially the case for us because nominals are document creation times,
of which there will necessarily be a finite number, all totally ordered,
which in turn will make checking accessibility more efficient.
Keeping quantification over first-order variables will not affect
decidability since we take the domain of quantification to be
objects that exist at a particular world,
that is at the time indicated by the nominal which picks out that world.
That is we use presentist quantification as opposed to eternalist quantification. 

\subsubsection{Termination of the Tableaux/Decision Procedure}

\citet*{bolander} seems like the clearest place to start from for termination
of the tableaux.  \citet*{norgela2007some} and \citet*{norgela2012decidability}
seem potentially useful but they discuss things in terms of sequents rather than tableaux.


\section{AMR Interpretation in Hybrid Logic}

\subsection{Examples}

\enumsentence{
  a. Carl submitted the forms and everyone will sign up again tomorrow.
  \\
  b.
  \\
  \scriptsize\texttt{(a / and\\
    \hspace*{0.5cm}:op1 (s / scope\\
    \hspace*{1.0cm}:pred (f / fill-out-03 :ongoing - :complete + :time (b / before :op1 (n / now))\\
    \hspace*{1.5cm}:ARG0 (p / person\\
    \hspace*{2.0cm}:name (n2 / name\\
    \hspace*{2.5cm}:op "Carl"))\\
    \hspace*{1.5cm}:ARG1 (f2 / form))\\
    \hspace*{1.0cm}:ARG0 p\\
    \hspace*{1.0cm}:ARG1 f2)\\
    \hspace*{0.5cm}:op2 (s2 / scope\\
    \hspace*{1.0cm}:pred (m / submit-01  :ongoing - :complete + :time (a2 / after :op1 n)\\
    \hspace*{1.5cm}:ARG0 (p2 / person\\
    \hspace*{2.0cm}:mod (a3 / all))\\
    \hspace*{1.5cm}:ARG1 f2)\\
    \hspace*{1.0cm}:ARG0 p2\\
    \hspace*{1.0cm}:ARG1 f2))}
  \\
c. It was impossible not to notice the license plate.
\\
d.
\\
\small\texttt{(s / scope\\
  \hspace*{1.0cm}:pred (p / possible-01\\
  \hspace*{1.5cm}:ARG0 (n / notice-01  :ongoing - :complete + :time (b / before :op1 (n2 / now))\\
  \hspace*{2.0cm}:polarity (n3 / not)\\
  \hspace*{2.0cm}:ARG1 (c /car)\\
  \hspace*{1.5cm}:polarity (n4 / not))\\
  \hspace*{1.0cm}:ARG0 n4\\
  \hspace*{1.0cm}:ARG1 p))}
}
\smallskip
NB: Will complete these translations in full.
\subsection{Extraction Steps}
With the chosen annotation, the root node
can consist of either a logical connective
(\verb|and|, \verb|or|, or \verb|cond|) linking two
AMR graphs, or a \verb|scope| node with its following
predicate and arguments.      

\subsection{General Extraction Algorithm}

\section{Future Work}

\subsection{$\downarrow$ and Quantification over Nominals}
Main points, at the cost of undecidability with adding $\downarrow$
some additional things can be done,
and at the cost of the integration of generalized quantifiers you can
ostensibly handle even things like habitual aspect.

\subsection{AMR Reentrancy and Non-Temoral Nominals}


There are some difficulties with maintaining the usual notion of possible
worlds being maximal with this idea, but there seems to be a direct sympathy between
the predicative core of an AMR sentence and in general reentrancy of the nodes with the
idea of a nominal as a ``point of view'' rather than the ``name'' of a world.
Maybe things like epistemic logic could be helpful here.
\subsection{Automated Inference and HTab}

HTab \citep{htab} provides an implementation of $\mathcal{H}(@, \textsf{A})$,
which does not natively provide a way to reason with $P$, $H$,
or first-order quantification.  The effort required in making the needed
changes to handle these remains to be determined.
\subsection{The Future of AMR and Parsing for Semantic Features}

To what extent can current AMR parsers extract the needed semantic features
to make full use of automated inference?  Of UMR, Dialogue-AMR,
and the AMR annotation variants we've used,
which logistically has the best outlook?
\section{Conclusion}


\bibliographystyle{acl_natbib}
\bibliography{references}
\end{document}
