


% 
% File acl2019.tex
%
%% Based on the style files for ACL 2018, NAACL 2018/19, which were
%% Based on the style files for ACL-2015, with some improvements
%%  taken from the NAACL-2016 style
%% Based on the style files for ACL-2014, which were, in turn,
%% based on ACL-2013, ACL-2012, ACL-2011, ACL-2010, ACL-IJCNLP-2009,
%% EACL-2009, IJCNLP-2008...
%% Based on the style files for EACL 2006 by 
%%e.agirre@ehu.es or Sergi.Balari@uab.es
%% and that of ACL 08 by Joakim Nivre and Noah Smith

\documentclass[11pt]{article}
%%\usepackage[hyperref]{acl2019}
\usepackage{times}
\usepackage{amsmath}
\usepackage{amssymb}
\usepackage{latexsym}
\usepackage{lingstyle}
\usepackage{graphicx}
\usepackage{tikz}
\usepackage{bussproofs}
\usepackage{natbib}
\usepackage{cite}
\usepackage[margin=0.5in]{geometry}
\usetikzlibrary{arrows,positioning,shapes} 
\tikzset{
    %Define standard arrow tip
    >=stealth',
    %Define style for boxes
    amrnode/.style={
             ellipse,
             draw=black, very thick,
             font=\small},
    scopenode/.style={
             ellipse,
             draw=red, very thick, dashed,
             text=red,
             font=\small},
    % Define arrow style
    amrarrow/.style={
            ->,
            thick,
            font=\small},
    scopearrow/.style={
              ->,
              red,
              thick,
              dashed,
              font=\small}
}



 \def\drs#1#2{\begin{tabular}{|l|}\hline #1 \\ \hline \\
                [-8pt] #2\\[-8pt] \\ \hline \end{tabular} }

 \def\ddrs#1{\begin{tabular}{||c||}\hline \\
                [-8pt] #1\\[-8pt] \\ \hline \end{tabular} }

 \def\topdrs#1#2{\begin{tabular}{|l|}\hline #1 \\ \hline \\
                [-8pt] #2 \\[-8pt] \\ \hline \end{tabular} }

 \def\proof#1#2#3#4{\drs{#1}{#2} \ $\vdash$ \ \drs{#3}{#4}}
 \def\modimp#1#2#3#4{\mbox{\drs{#1}{#2} \ $\Box$ \ \drs{#3}{#4}}}
 \def\imp#1#2#3#4{\drs{#1}{#2} \ $\Rightarrow$ \ \drs{#3}{#4}}
 \def\dis#1#2#3#4{\mbox{\drs{#1}{#2} \ $\vee$ \ \drs{#3}{#4}}}
 \def\int#1#2{\mbox{$^{\wedge}$ \ \drs{#1}{#2}}}
 \def\pos#1#2{\mbox{$\Diamond$ \ \drs{#1}{#2}}}
 \def\nec#1#2{\mbox{$\Box$ \ \drs{#1}{#2}}}
 \def\nega#1#2{\mbox{$\neg$ \ \drs{#1}{#2}}}
 \def\pred#1#2#3{\mbox{#1\ :\drs{#2}{#3}}}


\usepackage{url} 

\usepackage[acronym]{glossaries} % 'nomain' to disable automatic generation of "glossary" section 
\glsdisablehyper % disable hyperlink to non-existing glossary section 
%\aclfinalcopy % Uncomment this line for the final submission
%\def\aclpaperid{***} %  Enter the acl Paper ID here

%\setlength\titlebox{5cm}
% You can expand the titlebox if you need extra space
% to show all the authors. Please do not make the titlebox
% smaller than 5cm (the original size); we will check this
% in the camera-ready version and ask you to change it back.

\newcommand\BibTeX{B\textsc{ib}\TeX}

\title{Formalization of AMR Inference via Hybrid Logic Tableaux}
 

 
\author{
  Eli Goldner
}
\begin{document}
\maketitle
\begin{abstract}

  AMR and its extensions have become popular in semantic representation due
  to their ease of annotation by non-experts and
  attention to the predicative core of sentences, abstracting away from syntactic differences.
  An area where AMR and its extensions warrant improvement is formalization
  and suitability for inference, where it is lacking compared to
  other semantic/knowledge representations such as
  description logics, episodic logic, and discourse representation theory.
  This thesis presents a formalization of inference over AMR variants annotated for tense and aspect
  along with quantification and scope, via Blackburn and Marx's tableaux method for quantified hybrid
  logic, and Blackburn and Jørgensen's tableux method for basic hybrid tense logic.
  Hybrid logic's nominals are used to handle tense (and non-habitual aspect) in AMR via Blackburn's
  treatment of Reichenbach tenses for hybrid logic. Quantification, negation, and modality are
  handled natively in quantified hybrid logic.  We motivate the merging of these AMR variants,
  present their interpretation and inference in the combined quantified hybrid logic
  and basic hybrid tense logic, and demonstrate the soundess, completeness,
  and decidability of the combined logics. 

  
\end{abstract}


\section{Introduction}

\section{Related Work}

We draw from a number of areas which motivate this approach, namely
designing semantic representations for inferentiability,
the history and goals of AMR and its different annotations,
and hybrid logic with its variants and their accompanying tableaux methods for proof.  

\subsection{Inference in Semantic Representation}

Semantic representation is the task of representing meaning at the sentential
and potentially the discourse levels of language in a formally specifiable way.

\subsubsection{Discourse Representation Theory}

\subsubsection{Description Logics}

\subsubsection{Situation Semantics Influences -- Robin Cooper/Jonathan Ginzburg TTR Stuff?}


\subsubsection{Episodic Logic}


\subsection{Semantic Features in AMR and Possbility of Inference}

\subsection{Hybrid Logic and Our Chosen Semantic Features}

\section{Merging AMR Annotations}

\subsection{AMR Annotated for Tense and Aspect}

\subsection{AMR Annotated for Scope and Quantification}

\section{Merging Quantified Hybrid Logic and Indexical Hybrid Tense Logic}

\subsection{Background}

\subsubsection{Quantified Hybrid Logic}

\subsubsection{Indexical Hybrid Tense Logic}

\subsection{Quantified Hybrid Tense Logic}

\subsection{The Tableaux Calculus}

\subsection{Soundness and Completeness}
(Sketch)

\subsection{Decidability}
\subsubsection{Decidabililty of the Merged Logic}
\\
(Proof Sketch)
\\
While $\mathcal{H}(\downarrow @)$ is not decidable, $\mathcal{H}(@)$ is
\citep{Areces99hybridlogics}.  Quantified hybrid logic makes use of
$\downarrow$ \citep{quantified}, but modulo $\downarrow$ and quantification over
first order variables does not
differ from $\mathcal{H}(@)$.
Indexical hybrid tense logic does not make use of $\downarrow$
\citep{blackburn2012indexical}, and differs only from
$\mathcal{H}(@)$ in replacing the $\Box$ and $\Diamond$
with $F$ and $P$, which respectively have the same sematics
and similar semantics (the direction of the accesibility relation is changed)
to $\Diamond$.  Thus given the absence of $\downarrow$ in the merged logic,
permitting simultaneous standard use of $\Box$, $\Diamond$, $F$ and $P$,
will not have negatively affect the decidability given the analogous
complexity of $F$ and $P$ to $\Diamond$
(or equivalent complexity depending on how the inverse of accessibility relation between worlds is represented),
especially because for us nominals are document creation times,
of which there will necessarily be a finite number, all totally ordered,
which in turn will make checking accessibility more efficient.
Keeping quantification over first-order variables will not affect
decidability since we take the domain of quantification to be
objects that exist at a particular world,
that is at the time indicated by the nominal which picks out that world.
That is we use presentist quantification as opposed to eternalist quantification. 

\subsubsection{Decidability of the Tableaux/Decision Procedure}

\section{AMR Interpretation in Hybrid Logic}

\subsection{Examples}

\enumsentence{
  a. Carl submitted the forms and everyone will sign up again tomorrow.
  \\
  b.
  \\
  \scriptsize\texttt{(a / and\\
    \hspace*{0.5cm}:op1 (s / scope\\
    \hspace*{1.0cm}:pred (f / fill-out-03 :ongoing - :complete + :time (b / before :op1 (n / now))\\
    \hspace*{1.5cm}:ARG0 (p / person\\
    \hspace*{2.0cm}:name (n2 / name\\
    \hspace*{2.5cm}:op "Carl"))\\
    \hspace*{1.5cm}:ARG1 (f2 / form))\\
    \hspace*{1.0cm}:ARG0 p\\
    \hspace*{1.0cm}:ARG1 f2)\\
    \hspace*{0.5cm}:op2 (s2 / scope\\
    \hspace*{1.0cm}:pred (m / submit-01  :ongoing - :complete + :time (a2 / after :op1 n)\\
    \hspace*{1.5cm}:ARG0 (p2 / person\\
    \hspace*{2.0cm}:mod (a3 / all))\\
    \hspace*{1.5cm}:ARG1 f2)\\
    \hspace*{1.0cm}:ARG0 p2\\
    \hspace*{1.0cm}:ARG1 f2))}
  \\
c. It was impossible not to notice the license plate.
\\
d.
\\
\small\texttt{(s / scope\\
  \hspace*{1.0cm}:pred (p / possible-01\\
  \hspace*{1.5cm}:ARG0 (n / notice-01  :ongoing - :complete + :time (b / before :op1 (n2 / now))\\
  \hspace*{2.0cm}:polarity (n3 / not)\\
  \hspace*{2.0cm}:ARG1 (c /car)\\
  \hspace*{1.5cm}:polarity (n4 / not))\\
  \hspace*{1.0cm}:ARG0 n4\\
  \hspace*{1.0cm}:ARG1 p))}
}

\subsection{Extraction Steps}
With the chosen annotation the root node
can consist of either a logical connective
(\verb|and|, \verb|or|, or \verb|cond|) linking two
AMR graphs, or a \verb|scope| node with its following
predicate and arguments.      

\subsection{}

\section{Future Work}

\subsection{$\downarrow$ and Quantification over Nominals}

\subsection{AMR Reentrancy and Non-Temoral Nominals}

\subsection{Automated Inference and HTab}

\subsection{The Future of AMR and Parsing for Semantic Features}

\section{Conclusion}


\bibliographystyle{acl_natbib}
\bibliography{references}
\end{document}
